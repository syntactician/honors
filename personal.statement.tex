\documentclass[doc,12pt,natbib]{apa6}
\usepackage[colorlinks=false]{hyperref}
\usepackage{amssymb,amsmath,times}
\linespread{1.5}

\begin{document}

\title{Personal Statement}
\shorttitle{Personal Statement}
\author{Edward Hern\'{a}ndez}
\date{2 March 2016}
\affiliation{College of William \& Mary}
% \maketitle

In high school, I was a Policy debater. I debated on topics of space
exploration, transportation infrastructure, healthcare policy, and ethics.
Debate taught me about topics as various as public policy, history, economics,
philosophy, and literary criticism, many of which I would not have been exposed
to otherwise. More importantly, it taught me to reason, state and support
strong positions, and how to do my own research on a topic, finding,
evaluating, and synthesizing sources to inform my cases.

Despite this experience, when I first came to the College, I was nervous about
research. Orientation events told me it was important, and that I should do it,
but my idea of college-level research was based on my friends in the natural
sciences working with NASA and Jefferson Lab. I had no picture of research in
the social sciences or humanities. My first semester, I sought out and joined a
laboratory in the social sciences: the Social Networks and Political Psychology
(SNaPP) Laboratory in the Government Department. In the lab, I worked on
studies on political attitudes, contention, and disengagement. I learned about
the trend of increasing polarization and the political disengagement that this
inspires in many members of the public. Though my primary academic interests
turned toward Linguistics and Education, I have kept working in the SNaPP Lab,
and its focus on and insights into political disengagement have continued to
influence me. 

In my second semester of freshman year, I took Language Attitudes with Prof.
Anne Charity Hudley, and started working on the idea that would eventually
become this honors thesis. My initial research question was whether or not
speaking a non-standardized variety of English was detrimental to success in
debate. I had noticed, while I was in high school, that despite my school and
our district being quite diverse, our debaters were mostly White, and the
debaters that won were almost always White. While I began to investigate this
question, I also started volunteering as a judge for local high school debate
tournaments, endeavoring to judge fairly and without regard for whether or not
a student's language was standardized. My interest in these class topics of
language variation and ideologies inspired me to pursue a linguistics major.

In my sophomore year, I took African American English and Community Based
Research Methods with Prof. Charity Hudley, and continued to work on this
project. I soon realized that I could not change the culture of debate by
judging. Even if the problem was that judges were undervaluing arguments
delivered in non-Standardized language varieties, being one anomalous judge in
a bad pool would make little difference. The only way to influence judges'
attitudes and practices was to replace the judging pool or teach the existing
judges to value students' native language variations and cultural modes of
communication. At that point, I began plans to create a curriculum for training
judges that encouraged culturally responsive, equitable judging standards.

In my junior year, I began to serve as a Teaching Assistant for Prof. Charity
Hudley's classes, but continued to work with her on this topic. I remained
interested in creating a curriculum for judges, but I slowly began to realize
that it would be misguided to design curricula for judges without creating
matching curricula for students, to encourage them to embrace and celebrate
their language varieties and experiences. I began to read more literature in
Education, and stumbled across \citet{Banks15}, who discusses multicultural
education as a tool for empowering students toward civic engagement. I
immediately saw parallels with my experience with the disengagement literature
from the SNaPP Lab. Since then, I have been reading more Education literature,
and I have decided to pursue education as a career.

After I graduate, I intend to pursue a Masters in Education and to then teach
Secondary English. I will, without a doubt, implement debates in my Secondary
English classrooms, and I want to explore the relevant literature and design
curricula which will be useful to me and to other English educators who wish to
empower their underrepresented students to value their experience, their ideas,
and their language, and to be civically and politically engaged.

\clearpage
\bibliography{honors}

\end{document}
