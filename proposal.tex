\documentclass[man,12pt,natbib]{apa6}
\usepackage[colorlinks=false]{hyperref}
\usepackage{amssymb,amsmath,times}
\linespread{1.5}

\begin{document}

\title{Honors Thesis Proposal}
\shorttitle{Honors Thesis Proposal}
\author{Edward Hern\'{a}ndez}
\date{2 March 2016}
\affiliation{College of William \& Mary}
\maketitle

My research will investigate and design debate curricula for Secondary English
educators and classrooms to empower students to use their own voices that will
also train educators to value and respect those voices. I will be producing a
literature review and a set of secondary curricula based on that literature.

Debate as an educational practice has the potential to radically empower
students. Debate forces students to recontextualize topics and evidence in
light of new ideas and their experiences. A well-designed debate curriculum
could very easily affirm students' language varieties, cultural modes of
expression, rhetorics, and lived experience. However, in my experience, and
according to \citet{Polson12}, academic debate is less than welcoming of
diverse students and their language varieties. I argue that this is due in
large part to un- or under-trained educators applying the ``banking model'' of
education \citep{Friere68} and ideologies about language \citep{LippiGreen11}
to their debate curricula. My work will seek to address this problem using
insights from the Education and Literacy literatures to design curricula for
both students and educators.

The Multicultural Education \citep{} and Literacy Studies \citep{} literature .
. . Under a Literacy Framework, students are encouraged to recontextualize
course material in light of their lived experience. This approach validates the
student's experience and removes the educator as the sole epistemic authority.
The Multicultural Education literature challenges the teaching of Standardized
English, instead pushing us to value students' native language varieties and
the modes in which they are already adept at expressing themselves
\citep{NCTE}.

My primary methodologies will be literature review and curriculum design. I
will begin my research this summer by examining curricula in Secondary
English-relevant English classes, and delving deeply into the 

\clearpage
\bibliography{honors}

\end{document}
