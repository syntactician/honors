\documentclass[man,12pt,natbib]{apa6}
\usepackage[colorlinks=false]{hyperref}
\usepackage{amssymb,amsmath,times}
\linespread{1.5}

\begin{document}

\title{Honors Thesis Proposal}
\shorttitle{Honors Thesis Proposal}
\author{Edward Hern\'{a}ndez}
\date{2 March 2016}
\affiliation{College of William \& Mary}
\maketitle

My research will investigate and design debate curricula to empower Secondary
English students to use their own voices and train educators to value and
respect those voices. I will produce a literature review and a set of curricula
for Secondary English students and educators based on that literature.

Debate has the potential to radically empower students. It forces students to
recontextualize topics in light of new ideas and perspectives. A well-designed
debate curriculum can affirm student’s lived experience and cultural modes of
expression. However, in my experience, and according to \citet{Polson12},
academic debate is less than welcoming of diverse students and their language
varieties. I argue that this is due in large part to un- or under-trained
educators applying the ``banking model'' of education \citep{Friere68} and
Standardization ideologies about language \citep{LippiGreen11} to their debate
curricula. My work will seek to address this problem using insights from the
Education and Literacy literatures to design curricula for both students and
educators.

In creating these curricula, I will draw on the Multicultural Education
\citep{Banks93,Banks15} literature and the Literacy Studies
\citep{Street08,Street12} literatures. Under a Literacy framework, students'
preexisting skills are respected, and they are encouraged to recontextualize
course material in light of their lived experience. This model validates the
students’ experiences and abilities and removes the educator as the sole
epistemic authority. The Multicultural Education literature also challenges
teaching Standardized English, instead pushing us to value students’ native
language varieties, and the modes in which they are already adept at expressing
themselves \citep{NCTE}. Debate offers a vehicle to do both.

My primary methodologies will be literature review and curriculum design. I
will begin my research this summer by examining curricula in Secondary
English-relevant English classes, and delving deeply into literature related to
debate and multiculturalism in journals like the Journal of Multicultural
Education, Literacy in Compositional Studies, and Argumentation and advocacy.
This will give me a broad view of what is currently being published. This
summer I will produce a functional literature review. I expect that much of the
relevant Education and Forensic literature will suffer from the problems
discussed above, so I will not only report the literature, but also critique it
through a Critical Race Theory lens. This review will culminate in
recommendations for culturally responsive curricular design specifically
utilizing debate.

Starting in the Fall 2016 semester and continuing in the Spring 2017, I will,
using the recommendations I have gleaned from the literature, begin designing
two distinct but interconnected curricula, students and educators, one for
students and one for educators. These curricula will contain materials and
plans not only to facilitate debate in the secondary English classroom, but
also to train educators to judge debate in fair, culturally responsive ways
that value students’ experiences and modes of expression.

In high school, I competed in Lincoln-Douglas, Public Forum and Policy Debate.
These experiences were crucial to my development as a student and as a person,
and they taught me about topics from global politics to literary theory. More
importantly, they taught me how to reason critically and to research. I have
frequently returned to my high school district, volunteering as a judge at
various tournaments and teaching new debaters at several schools, in order to
offer the same opportunities to others.

I have also worked on these ideas with Prof. Anne Charity Hudley since my
freshman year. My freshman year, in her Language Attitudes course, I researched
debate literature to understand judging and assessment norms in debate
tournaments. During my sophomore year, I took her African American Language and
Community-Based Research Methods courses, focusing on racialized attitudes
toward variations of English and the impacts they have on debate. This year, I
served as a Teaching Assistant for her classes as I planned this thesis and
helped other students with related projects. In the last year, I have also
attended School-University Research Network events with Prof. Charity Hudley,
and begun to discuss my project with current teachers of English.

My work is needed because it bridges a gap between the English and Education
literatures, which are otherwise fragmented and draw on separate theories and
terminology.  I believe, however, that the Multicultural Education and
Literacies literatures are consistent in spirit, and that my project is poised
to draw on both and meaningfully combine them. Further, I think that the
majority of literacy work focuses almost the education of students, so my work
will be unusual in that I want to create parallel educational curricula for
both students and educators.

In order to pursue this project, I need coursework in English and Education,
access to journals, and to be free to work on this project. None of those come
for free. An honors fellowship would allow me to pay for access to Education
and Forensics journals to which the College is not subscribed, pay for my
English Education classes, and keep me from having to work this summer, leaving
me free to work on my honors thesis.

\clearpage
\bibliography{honors}

\end{document}
