\documentclass[man,12pt,natbib]{apa6}
\usepackage[colorlinks=false]{hyperref}
\usepackage{amssymb,amsmath,times}
\linespread{1.5}

\begin{document}

\title{Honors Thesis Proposal}
\shorttitle{Honors Thesis Proposal}
\author{Edward Hern\'{a}ndez}
\date{2 March 2016}
\affiliation{College of William \& Mary}
\maketitle

% What is the question that you hope to answer with your research?
My research seeks to understand how we can better empower students to use their
own voices and train educators to value and respect those voices.

Students have a wide variety of lived experiences, and they express themselves
in various ways. They speak many different languages, and countless varieties
of each of those languages. Often, English education devalues these various
modes of expression and communication, viewing them as inferior to Standard or
Standardized English. In this mindset, school is conducted in Standardized
English, and one of its goals is to teach this Standardized variety. Teachers
in this model claim authority over language \citep{LippiGreen11} and the course
material, and merely deposit their knowledge into students \citep{Friere68}.

The Multicultural Education literature and the Literacy Studies literature
challenge these modes  . . .

Debate as an educational practice also radically challenges this form of
education. Like the Education and Literacy literatures suggest . . .

% What is the methodology that you will employ to pursue this project?
My primary methodologies will be literature review and curriculum design.

% What will your research contribute to the field and why is this work exciting
% or unique?
I believe that the English and Education literatures are fragmented. Due to
their long history as fields, they draw on separate theories and terminology. I
also believe that the ``multicultural education'' and ``literacies''
literatures are consistent in spirit, and that my project is poised to draw on
both and meaningfully combine them.
Further, I think that the majority of education and literacy work focuses
almost exclusively on the education of students. I want to create parallel
educational curricula for both students and educators.

% What course work or other experience have you had that are relevant to this
% proposal?

In high school, I was a debater, competing in Lincoln-Douglas, Public Forum and
Policy Debate. These experiences were crucial to my development as a student
and as a person, and they taught me about topics ranging from global economic
politics to critical literary theory. More importantly, they taught me how to
reason critically, to state and defend my position, and to research. However,
in my experience, local debate is not particularly welcoming of diverse
students or voices. Since high school, in an effort to help more and more
diverse students benefit from debate like I did, I have frequently returned to
my high school district, volunteering as a judge at various tournaments and
teaching new debaters at several schools. 

I have also worked with Prof. Anne Charity Hudley since my freshman year on
ideas related to this topic. 
My freshman year, in her Language Attitudes course, I researched debate
literature to understand norms around judging and assessment in debate
tournaments. During my sophomore year, I took her African American Language
course, in which I focused on attitudes toward racialized variations of English and the impacts that they might have on
debaters and judges, and Community-Based Research Methods, where I began to
design a project to tackle these subjects. This year, I have served as a
Teaching Assistant for her classes, continuing planning to begin this project
as an honors thesis and helping other students do the same with their related
projects.

% What support do you need to pursue this project (e.g. lab space, access to
% archives or interview subjects, etc.)?
In order to pursue this project, I need coursework in English and Education,
access to Education and Forensics journals, and to be free to work on this
project. None of those come for free. An honors fellowship would allow me to
pay for access to niche literature, visit relevant special collections, and
travel to meet with faculty who do similar work at other institutions. It would
also fund my starting to take English Education classes over this summer, and
would keep me from having to work this summer to support myself, leaving me
free to work on my honors thesis.

% Plan for the next year.
If I receive this fellowship, I will use some of the funds to pay for a light
load of Education-relevant English courses over the summer, to pave the way for an Education degree after I graduate from the college. Over the summer, both 
in these classes and outside of them, I will extensively read English and
Education literature, producing a functional literature review as a starting
place for my thesis. As I near the end of the summer and begin the Fall
semester, I will begin designing curricula based on what I have gleaned from
the literature. I will aim to produce two distinct but interconnected
curricula, one for students and one for educators, to facilitate not just
student involvement, but also engagement and support on the part of educators.
In the Fall, I intend to survey current debate coaches and English teachers
about what materials they would want or need in order to engage more and more
diverse students or to implement debate-related curricula in the classroom.
This information will guide the creation of curricula in the Fall and Spring.

\clearpage
\bibliography{honors}

\end{document}
