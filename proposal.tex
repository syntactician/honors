\documentclass[man,12pt,natbib]{apa6}
\usepackage[colorlinks=false]{hyperref}
\usepackage{amssymb,amsmath,times}
\linespread{1.5}

\begin{document}

\title{Honors Thesis Proposal}
\shorttitle{Honors Thesis Proposal}
\author{Edward Hern\'{a}ndez}
\date{2 March 2016}
\affiliation{College of William \& Mary}
\maketitle

% What is the question that you hope to answer with your research?
My research seeks to understand how we can better empower students to use their
own voices and train educators to value and respect those voices.

Students have a wide variety of lived experiences, and they express themselves
in various ways. They speak many different languages, and countless varieties
of each of those languages. Often, English education devalues these various
modes of expression and communication, viewing them as inferior to Standard or
Standardized English. In this mindset, school is conducted in Standardized
English, and one of its goals is to teach this Standardized variety. Teachers
in this model claim authority over language \citep{LippiGreen11} and the course
material, and merely deposit their knowledge into students \citep{Friere68}.
This course material and the way in which it is taught may be largely
irrelevant to the lived experience of the students.

The Multicultural Education literature and the Literacy Studies literature
challenge this ``banking'' mode of education. Under a Literacy work, students
are encouraged to recontextualize course material in light of their lived
experience, validating the students' experiences and abilities and removing the
educator as the sole epistemic authority. The Multicultural Education
literature also challenges teaching Standardized English, instead pushing us to
value students' native language varieties, and the modes in which they are
already adept at expressing themselves.

Debate as an educational practice has the potential to radically affirm the
tenets of both of these literatures. Debate forces students to recontextualize
topics and evidence in light of new ideas and experiences, and prompts them to
express their ideas in novel contexts. A well designed debate curriculum could
very easily affirm student's language varieties, cultural modes of expression,
rhetorics, and lived experience. Debate is antithetical to the banking model,
as it prompts students to take ownership and authority over their ideas.
However, in my experience, academic debate is less than welcoming of diverse
students and language varieties. I argue that this is due in large part to
un- or under-trained educators attempting to extend the banking model and 
their Standardization ideologies to their debate curricula. My work will seek
to combat this problem by using insights from the Education and Literacy
literatures to design curricula for both students and educators.

% What is the methodology that you will employ to pursue this project?
% Plan for the next year.
My primary methodologies will be literature review and curriculum design.
I will begin my research this summer by taking a light load of
education-relevant English classes, and delving deeply into literature in
journals targeted at Education, Literacy Studies, and Forensics. This will give
me a broad understanding of what is being published in my topic area. Over this
summer I will produce a functional literature review. I expect that much of the
relevant Education literature and most of the Forensic literature will suffer
from some of the problems discussed above, so I will focus on not only
reporting the content of the literature, but critiquing it through a Critical
Race Theory lens. My literature review will culminate in recommendations for
culturally responsive curricular design utilizing debate as a tool to prompt
students to recontextualize and value their lived experiences and modes of
expression.

Starting in the Fall semester and continuing in the Spring, I will, using the
recommendations I have gleaned from the literature, begin designing two
distinct but interconnected curricula, students and educators, one for students
and one for educators. These curricula will contain materials and plans not
only to facilitate debate in the secondary English classroom, but also to train
educators to judge debate in fair, culturally responsive ways that value
students' experiences and modes of expression.

% What course work or other experience have you had that are relevant to this
% proposal?

In high school, I was a debater, competing in Lincoln-Douglas, Public Forum and
Policy Debate. These experiences were crucial to my development as a student
and as a person, and they taught me about topics ranging from global economic
politics to critical literary theory. More importantly, they taught me how to
reason critically, to state and defend my position, and to research. However,
in my experience, local debate is not particularly welcoming of diverse
students or voices. Since high school, in an effort to help more and more
diverse students benefit from debate like I did, I have frequently returned to
my high school district, volunteering as a judge at various tournaments and
teaching new debaters at several schools. 

I have also worked with Prof. Anne Charity Hudley since my freshman year on
ideas related to this topic. 
My freshman year, in her Language Attitudes course, I researched debate
literature to understand norms around judging and assessment in debate
tournaments. During my sophomore year, I took her African American Language
course, in which I focused on attitudes toward racialized variations of English and the impacts that they might have on
debaters and judges, and Community-Based Research Methods, where I began to
design a project to tackle these subjects. This year, I have served as a
Teaching Assistant for her classes, continuing planning to begin this project
as an honors thesis and helping other students do the same with their related
projects.  For the last year, I have also attended SURN events with Prof.
Charity Hudley, and begun to discuss my ideas about debate curricula with
current teachers of English.

% What will your research contribute to the field and why is this work exciting
% or unique?
I believe that my work is needed, because the English and Education literatures
are fragmented. Due to their long histories as fields, they draw on separate
theories and terminology. Literacy Studies stems from English literary
criticism, and draws heavily on critical frameworks. The Education literature
is pragmatic, and draws on its own body of theory. I do believe, however, that
the ``multicultural education'' and ``literacies'' literatures are consistent
in spirit, and that my project is poised to draw on both and meaningfully
combine them.  Further, I think that the majority of literacy work focuses
almost the education of students, so my work will be unusual in that I want to
create parallel educational curricula for both students and educators.

% What support do you need to pursue this project (e.g. lab space, access to
% archives or interview subjects, etc.)?
In order to pursue this project, I need coursework in English and Education,
access to Education and Forensics journals, and to be free to work on this
project. None of those come for free. An honors fellowship would allow me to
pay for access to niche literature, visit relevant special collections, and
travel to meet with faculty who do similar work at other institutions. It would
also fund my starting to take English Education classes over this summer, and
would keep me from having to work this summer to support myself, leaving me
free to work on my honors thesis.


\clearpage
\bibliography{honors}

\end{document}
