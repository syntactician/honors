\documentclass[doc,12pt]{apa6}
\usepackage[colorlinks=false]{hyperref}
\usepackage{amssymb,amsmath,times}
\linespread{1.5}

\begin{document}

\title{Recommendation Letter}
\shorttitle{Recommendation Letter}
\author{Anne Charity Hudley}
\date{2 March 2016}
\affiliation{College of William \& Mary}
\maketitle

I have known Edward Hernandez since his freshman year at the College. I was his
pre-major adviser and I am now his adviser for both his Linguistics major and
his honors thesis. He has either taken or served as a TA for each of my classes
since Spring 2014. Outside of class, he has also consistently attended WMSURE
and met with me in office hours.

Edward got involved with research in the Social Networks and Political
Psychology (SNaPP) Laboratory, where he now serves as Lab Director, during his
first semester at the College. In that lab, he has designed, implemented,
proctored, coded, and analyzed experiments in behavioral political science,
focusing on questions of political polarization, contention, and disengagement.

In his second semester, he began working with me in class on the ideas which
developed into his honors thesis. In Spring 2014, he began working with me on
issues of language standardization in debate. His idea was that students who
spoke non-standardized varieties of English were received poorly in the local
debate scene, and this constituted a barrier to entry to debate, which he
viewed as a valuable educational resource. In that semester, he also began to
judge high school debate tournaments in his home district, volunteering his
time to facilitate the  operations of the understaffed local leagues. He has
continued to volunteer in this capacity, as well as teaching new debaters at
several schools and training several new coaches and judges in Newport News and
Williamsburg.

His work on debate has expanded steadily over the last two years, and has
evolved from a  Linguistics project on the language varieties of student
debaters to an interdisciplinary project encompassing Education, English, and
Linguistics literatures, as well as methodologies from Community Studies. His
close, continued work with the local debate community has uniquely prepared him
to design and implement debate curricula, as he is both immersed in the
academic literature and still integrated in the community he wishes to serve
with this project.

Outside of my classes, Edward has served as a Research Assistant to Professor
Cheryl Dickter in the Psychology department, working to implement, proctor, and
code psychological and neuroscientific experiments to understand, measure, and
reduce racism and homophobia. He is currently working with me and Professor
Dickter on an upcoming project merging our research interests. He also works as
a Research Assistant to Professor Leslie Cochrane, studying sociolinguistic
variation among speakers with physical disabilities.

After he graduates, Edward plans to pursue a Masters in Education, and he
intends to teach secondary English. His honors project, with its focus on
educational debate and curricular design, will immerse him in the literature he
needs for grad school and will give him a head start on the work that he will
be doing, designing and implementing curricula for the classroom. The honors
fellowship will support him over the next summer, which will allow him to work
on his honors thesis instead of seeking full-time employment to support
himself.

\end{document}
